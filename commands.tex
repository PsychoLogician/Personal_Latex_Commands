% START A DOCUMENT
% \documentclass{article} %book report

% % Commands and abbreviations
%   \usepackage{
    amsmath, amsthm, amssymb,
    comment, %for the ※ symbol
    hyperref, %for innerlinks in a file 
    bookmark, %for BOOKMARKS
    mathtools,
    siunitx,
    esvect,
    bm,
    extarrows, %see https://ctan.yazd.ac.ir/macros/latex/contrib/extarrows/extarrows-test.pdf
    import, % \import{./}{filename.tex}
    cancel,
        % $\cancelto{2}{1}$ 
        % documentation: https://mirror.las.iastate.edu/tex-archive/macros/latex/contrib/cancel/cancel.pdf
    tikz-cd, 
        % diagrams and categories. 
        % https://ctan.math.washington.edu/tex-archive/graphics/pgf/contrib/tikz-cd/tikz-cd-doc.pdf
    %
    setspace, 
    }
    
\usetikzlibrary{decorations.pathmorphing} %for rightsquigarrow in tikzcd


\newcommand{\diag}[1]{\rotatebox[origin=c]{45}{#1}}
\newcommand{\dig}[1]{\begin{tikzcd}[ampersand replacement=\&] #1 \end{tikzcd}} 
\newcommand{\new}[1]{\textcolor{blue}{#1}}
\newcommand{\Prod}[3]{\displaystyle \prod_{#1}^{#2}#3}
\newcommand{\pair}[2]{\langle #1,#2\rangle}
\newcommand{\cpair}[2]{( #1,#2 )}
\newcommand{\gcode}[1]{\ulcorner #1 \urcorner}
\newcommand{\st}{\, s.t. \,}
\newcommand{\iiff}{\leftrightarrow}
\renewcommand{\j}{\cat{J}}
\renewcommand{\i}{\cat{I}}

% local
  \newcommand{\cat}[1]{\mathbf{#1}}
  \newcommand{\pull}[3]{p_{#1}(#2,#3)}
  \newcommand{\prd}[2]{<#1,#2>}
  % \renewcommand{\lim}[2]{\underset{\underset{#2}{\leftarrow}}{lim} #1_{#2}}
  \newcommand{\colim}[2]{\underset{\underset{#2}{\rightarrow}}{lim} #1_{#2}}

  \newcommand{\rest}[1]{\hspace{-3pt}\restriction _{#1}}
  % \newcommand{\mor}[3]{#1 \xlongrightarrow{#2} \raisebox{0.57ex}{\text{$\xlongrightarrow{#2}$}}  #3}
  \newcommand{\mor}[3]{#1 \xlongrightarrow{#2} #3}
  % \newcommand{\twomor}[4]{
  %   \begin{tikzcd}[ampersand replacement=\&]
  %     \& #1 \arrow[r,"#2"] \arrow[r,"#3" below,yshift=0.7ex] \& #4 
  %   \end{tikzcd}
  % }
% special symbols
  \newcommand{\cont}[0]{\text{\textreferencemark}}
  \newcommand{\0}{\emptyset}
  \newcommand{\nat}{\mathbb{N}}
  \newcommand{\vf}{\varphi}
% 

% arrows
  \newcommand{\ar}{\longrightarrow}
  \newcommand{\arr}{\Longrightarrow}
  \newcommand{\xar}[1]{\xlongrightarrow{#1}}
  \newcommand{\xart}[1]{\xlongrightarrow{\text{#1}}}
  \newcommand{\xarr}[1]{\xLongrightarrow{#1}}
  \newcommand{\xarrt}[1]{\xLongrightarrow{\text{#1}}}
  \newcommand{\xeqt}[1]{\xlongequal{\text{#1}}}
  \newcommand{\xeq}[1]{\xlongequal{#1}}

% New theorems 
  \newtheorem{lemma}{Lemma}
  \newtheorem*{lemma*}{Lemma}
  
  \newtheorem{proposition}{Proposition}
  \newtheorem*{proposition*}{Proposition}
  
  \newtheorem{thm}{Theorem}
  \newtheorem*{thm*}{Theorem}

  \newtheorem{remark}{Remark}

  \newtheorem*{def*}{Definiton}
  % \newtheorem*{arb*}{}

% 

% quantifiers
  \newcommand{\ex}[1]{\exists_{#1}\,}
  \newcommand{\all}[1]{\forall_{#1}\,}
% 

% abbreviations
  \newcommand{\set}[1]{\{{#1}\}}
  \newcommand{\func}[3]{#1: #2 \longrightarrow #3}
  \newcommand{\funcc}[5]{
        \begingroup
        \setlength\arraycolsep{0pt}
            \begin{array}[t]{ll}
                #1: & #2 \to #3 \\ 
                    & #4 \mapsto #5 
            \end{array}
        \endgroup
    }
  \newcommand{\andd}{\, \& \,}
  \newcommand{\orr}{\, or \,}
  \newcommand{\seq}[1]{#1_{0},#1_{1},\dots }
  \newcommand{\oseq}[1]{#1_{0}\leq #1_{1}\leq \dots }
% 


% tabbing and indention
  \newcommand{\ind}[1]{\text{} %\vspace{-10pt}
    \begin{enumerate}%[leftmargin=15pt] %default was about 25pt
        \begin{onehalfspace}  %doublespace, onehalfspace   % https://t.me/c/1752359800/915
            #1   
        \end{onehalfspace}
    \end{enumerate}
    \vspace{-15pt}
  }
  \newcommand{\tb}[1]{\text{} %\vspace{-10pt}
  \begin{itemize}%[leftmargin=15pt] %default was about 25pt
      \begin{onehalfspace}  %doublespace, onehalfspace   % https://t.me/c/1752359800/915
          #1   
      \end{onehalfspace}
  \end{itemize}
  \vspace{-15pt}
}
  \newcommand{\mnl}[1]{\tb{\item[] $#1$}}
  \newcommand{\mnlt}[1]{\tb{\item[] #1}}

% 

% Category
  % prototypes
    % exponentiation diagram
    % product diagrams 
      \newcommand{\catcomma}{\text{$,$}}
      \newcommand{\catexp}[4]{
        \begin{tikzcd}[ampersand replacement=\&]
          \& {#1}^{#2}  \& {#1}^{#2}\times {#2}  \arrow[r,"\epsilon"]   \& {#1}\\
          \&  #3 \arrow[u,"\hat{#4}"]    \& #3 \times {#2}  \arrow[u,"\hat{#4} \times 1_{#2} "] \arrow[ur,"#4" below]  \&
        \end{tikzcd}
      }

      \newcommand{\catexprev}[4]{
        \begin{tikzcd}[ampersand replacement=\&]
          \& {#1}^{#2}  \&  {#2} \times {#1}^{#2}  \arrow[r,"\epsilon"]   \& {#1}\\
          \&  #3 \arrow[u,"\widehat{#4}"]    \& #2 \times #3  \arrow[u," 1_{#2} \times \widehat{#4} "] \arrow[ur,"#4" below]  \&
        \end{tikzcd}
      }

      \newcommand{\catprod}[5]{
        \begin{tikzcd}[ampersand replacement=\&]
          \&           \& #3  \arrow[dl,"#4" above] \arrow[dr,"#5" ] \arrow[d, "\langle #4 \text{$,$} #5 \rangle",dotted ] \& \\
          \& #1        \& #1 \times #2 \arrow[l,"\pi_{1}" ] \arrow[r,"\pi_{2}" below ]                                 \&   #2
        \end{tikzcd}
      }
      \newcommand{\diagram}[3]{#1: \mathbf{#2} \to #3}
      \newcommand{\parallelsquare}[8]{
        \begin{tikzcd}[ampersand replacement=\&]
          \& #1 \arrow[r,"#5"] \arrow[d,"#6"]        \& #2  \arrow[d,"#7"] \\
          \& #3 \arrow[r,"#8"]                       \& #4 
        \end{tikzcd}
      }
      \newcommand{\ntsquare}[8]{
        \begin{tikzcd}[ampersand replacement=\&]
          #1 \arrow[r,"#5"] \arrow[d,"#6" description]        \& #2  \arrow[d,"#7"] \arrow[dl,"\rotatebox{45}{=}",phantom]\\
          #3 \arrow[r,"#8"]                       \& #4 
        \end{tikzcd}
      }
      \newcommand{\simplediag}[1]{\begin{tikzcd}[ampersand replacement=\&] #1 \end{tikzcd}}

% Localizing commands  of separate files
  % A new null environment
    \newenvironment{nullenv}{}{} %https://www.overleaf.com/learn/latex/Environments
  % Command for localization
    \newcommand{\myinput}[1]{\nullenv{\input{#1}}}


% % If Persian...
%   % \usepackage{xepersian} \settextfont{BNazanin.ttf}
\usepackage{
    amsmath, amsthm, amssymb,
    comment, %for the ※ symbol
    hyperref, %for innerlinks in a file 
    bookmark, %for BOOKMARKS
    mathtools,
    siunitx,
    esvect,
    bm,
    extarrows, %see https://ctan.yazd.ac.ir/macros/latex/contrib/extarrows/extarrows-test.pdf
    import, % \import{./}{filename.tex}
    cancel,
        % $\cancelto{2}{1}$ 
        % documentation: https://mirror.las.iastate.edu/tex-archive/macros/latex/contrib/cancel/cancel.pdf
    tikz-cd, 
        % diagrams and categories. 
        % https://ctan.math.washington.edu/tex-archive/graphics/pgf/contrib/tikz-cd/tikz-cd-doc.pdf
    %
    setspace, 
    }
    
\usetikzlibrary{decorations.pathmorphing} %for rightsquigarrow in tikzcd

\newcommand{\diag}[1]{\rotatebox[origin=c]{45}{#1}}
\newcommand{\dig}[1]{\begin{tikzcd}[ampersand replacement=\&] #1 \end{tikzcd}} 
\newcommand{\new}[1]{\textcolor{blue}{#1}}
\newcommand{\Prod}[3]{\displaystyle \prod_{#1}^{#2}#3}
\newcommand{\pair}[2]{\langle #1,#2\rangle}
\newcommand{\cpair}[2]{( #1,#2 )}
\newcommand{\gcode}[1]{\ulcorner #1 \urcorner}
% \newcommand{\st}{\, s.t. \,}
\newcommand{\iiff}{\leftrightarrow}
\renewcommand{\j}{\cat{J}}
\renewcommand{\i}{\cat{I}}

% local
  \newcommand{\cat}[1]{\mathbf{#1}}
  \newcommand{\pull}[3]{p_{#1}(#2,#3)}
  \newcommand{\prd}[2]{<#1,#2>}
  % \renewcommand{\lim}[2]{\underset{\underset{#2}{\leftarrow}}{lim} #1_{#2}}
  \newcommand{\colim}[2]{\underset{\underset{#2}{\rightarrow}}{lim} #1_{#2}}

  \newcommand{\rest}[1]{\hspace{-3pt}\restriction _{#1}}
  % \newcommand{\mor}[3]{#1 \xlongrightarrow{#2} \raisebox{0.57ex}{\text{$\xlongrightarrow{#2}$}}  #3}
  \newcommand{\mor}[3]{#1 \xlongrightarrow{#2} #3}
  % \newcommand{\twomor}[4]{
  %   \begin{tikzcd}[ampersand replacement=\&]
  %     \& #1 \arrow[r,"#2"] \arrow[r,"#3" below,yshift=0.7ex] \& #4 
  %   \end{tikzcd}
  % }
% special symbols
  \newcommand{\cont}[0]{\text{\textreferencemark}}
  \newcommand{\0}{\emptyset}
  \newcommand{\nat}{\mathbb{N}}
  \newcommand{\vf}{\varphi}
% 

% arrows
  \newcommand{\ar}{\longrightarrow}
  \newcommand{\arr}{\Longrightarrow}
  \newcommand{\xar}[1]{\xlongrightarrow{#1}}
  \newcommand{\xart}[1]{\xlongrightarrow{\text{#1}}}
  \newcommand{\xarr}[1]{\xLongrightarrow{#1}}
  \newcommand{\xarrt}[1]{\xLongrightarrow{\text{#1}}}
  \newcommand{\xeqt}[1]{\xlongequal{\text{#1}}}
  \newcommand{\xeq}[1]{\xlongequal{#1}}

% New theorems 
  \newtheorem{lemma}{Lemma}
  \newtheorem*{lemma*}{Lemma}
  \newtheorem{plem}{لم}
  \newtheorem*{plem*}{لم}

  \newtheorem{proposition}{Proposition}
  \newtheorem*{proposition*}{Proposition}
  
  \newtheorem{thm}{Theorem}
  \newtheorem*{thm*}{Theorem}
  \newtheorem{pthm}{قضیه}
  \newtheorem*{pthm*}{قضیه}

  \newtheorem{remark}{Remark}

  \newtheorem*{def*}{Definiton}
  % \newtheorem*{arb*}{}

% 

% quantifiers
  \newcommand{\ex}[1]{\exists_{#1}\,}
  \newcommand{\all}[1]{\forall_{#1}\,}
% 

% abbreviations
  \newcommand{\set}[1]{\{{#1}\}}
  \newcommand{\func}[3]{#1: #2 \longrightarrow #3}
  \newcommand{\funcc}[5]{
        \begingroup
        \setlength\arraycolsep{0pt}
            \begin{array}[t]{ll}
                #1: & #2 \to #3 \\ 
                    & #4 \mapsto #5 
            \end{array}
        \endgroup
    }
  \newcommand{\andd}{\, \& \,}
  \newcommand{\orr}{\, or \,}
  \newcommand{\seq}[1]{#1_{0},#1_{1},\dots }
  \newcommand{\oseq}[1]{#1_{0}\leq #1_{1}\leq \dots }
% 


% tabbing and indention
  \newcommand{\ind}[1]{\text{} %\vspace{-10pt}
    \begin{enumerate}%[leftmargin=15pt] %default was about 25pt
        \begin{onehalfspace}  %doublespace, onehalfspace   % https://t.me/c/1752359800/915
            #1   
        \end{onehalfspace}
    \end{enumerate}
    \vspace{-15pt}
  }
  \newcommand{\tb}[1]{\text{} %\vspace{-10pt}
  \begin{itemize}%[leftmargin=15pt] %default was about 25pt
      \begin{onehalfspace}  %doublespace, onehalfspace   % https://t.me/c/1752359800/915
          #1   
      \end{onehalfspace}
  \end{itemize}
  \vspace{-15pt}
}
  \newcommand{\mnl}[1]{\tb{\item[] $#1$}}
  \newcommand{\mnlt}[1]{\tb{\item[] #1}}

% 

% Category
  % prototypes
    % exponentiation diagram
    % product diagrams 
      \newcommand{\catcomma}{\text{$,$}}
      \newcommand{\catexp}[4]{
        \begin{tikzcd}[ampersand replacement=\&]
          \& {#1}^{#2}  \& {#1}^{#2}\times {#2}  \arrow[r,"\epsilon"]   \& {#1}\\
          \&  #3 \arrow[u,"\hat{#4}"]    \& #3 \times {#2}  \arrow[u,"\hat{#4} \times 1_{#2} "] \arrow[ur,"#4" below]  \&
        \end{tikzcd}
      }

      \newcommand{\catexprev}[4]{
        \begin{tikzcd}[ampersand replacement=\&]
          \& {#1}^{#2}  \&  {#2} \times {#1}^{#2}  \arrow[r,"\epsilon"]   \& {#1}\\
          \&  #3 \arrow[u,"\widehat{#4}"]    \& #2 \times #3  \arrow[u," 1_{#2} \times \widehat{#4} "] \arrow[ur,"#4" below]  \&
        \end{tikzcd}
      }

      \newcommand{\catprod}[5]{
        \begin{tikzcd}[ampersand replacement=\&]
          \&           \& #3  \arrow[dl,"#4" above] \arrow[dr,"#5" ] \arrow[d, "\langle #4 \text{$,$} #5 \rangle",dotted ] \& \\
          \& #1        \& #1 \times #2 \arrow[l,"\pi_{1}" ] \arrow[r,"\pi_{2}" below ]                                 \&   #2
        \end{tikzcd}
      }
      \newcommand{\diagram}[3]{#1: \mathbf{#2} \to #3}
      \newcommand{\parallelsquare}[8]{
        \begin{tikzcd}[ampersand replacement=\&]
          \& #1 \arrow[r,"#5"] \arrow[d,"#6"]        \& #2  \arrow[d,"#7"] \\
          \& #3 \arrow[r,"#8"]                       \& #4 
        \end{tikzcd}
      }
      \newcommand{\ntsquare}[8]{
        \begin{tikzcd}[ampersand replacement=\&]
          #1 \arrow[r,"#5"] \arrow[d,"#6" description]        \& #2  \arrow[d,"#7"] \arrow[dl,"\rotatebox{45}{=}",phantom]\\
          #3 \arrow[r,"#8"]                       \& #4 
        \end{tikzcd}
      }
      \newcommand{\simplediag}[1]{\begin{tikzcd}[ampersand replacement=\&] #1 \end{tikzcd}}

% Localizing commands  of separate files
  % A new null environment
    \newenvironment{nullenv}{}{} %https://www.overleaf.com/learn/latex/Environments
  % Command for localization
    \newcommand{\myinput}[1]{\nullenv{\input{#1}}}
